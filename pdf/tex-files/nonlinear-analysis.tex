
\section{Nonlinear Analysis}


\subsection{SSRT and UTDE}
%%%%%%%%%%%%%%%%%%%%%%%%%%%%%%%%%%%%%%%%%%%%%%%%%%%%%%%%
{
\paper{Takens F 1981 in {\bf Dynamical Systems and Turbulence}; Casdagli 1991 in {\bf Physica D}}

\begin{frame}{State Space Reconstruction Theorem}
    \begin{figure}
        \includegraphicscopyright[width=0.6\linewidth]{nonlinear-analyses/rss}
	{Figure is adapted from (Casdagli et al. 1991; Quintana-Duque (2012); Uzal et al. 2011)} 
	%\caption{Figure adapted from (Casdagli et al. 1991; Quintana-Duque (2012); Uzal et al. 2011)} 
   \end{figure}
	
\end{frame}
}


\subsection{SSRT and UTDE}
%%%%%%%%%%%%%%%%%%%%%%%%%%%%%%%%%%%%%%%%%%%%%%%%%%%%%%%%
{
\paper{Takens F 1981 in {\bf Dynamical Systems and Turbulence}; Casdagli 1991 in {\bf Physica D}}

\begin{frame}{Takens's Theorem}

\LARGE
%%********************************[EQUATION]************************************
\begin{equation*}\label{eq:measurement}
	s(t)= f^{t}[ s(0) ]
	%x(t)=h[ f^{t}s(0)  ]
\end{equation*}
%%********************************[EQUATION]************************************
%\vspace{0.1mm}
\normalsize
\begin{itemize}
\item $s$ represents a trajectory which evolves in an unknown $d-$dimensional manifold $M$
\item $f^t$ is a evolution function with time evolution $t$
\end{itemize}
Then 
\LARGE
%%********************************[EQUATION]************************************
\begin{equation*}\label{eq:measurement}
	x(t)=h[s(t)]
\end{equation*}
%%********************************[EQUATION]************************************
%\vspace{0.1mm}
\normalsize
\begin{itemize}
\item $x(t)$ scalar time series in $\mathbb{R}$ 
\item $h$ is a function defined on the trajectory $s(t)$
\end{itemize}


%where $h$ is a function, $h: M \rightarrow \mathbb{R}$, defined
%on the evolution function $f^t$ amd
	
\end{frame}
}


\subsection{}
%%%%%%%%%%%%%%%%%%%%%%%%%%%%%%%%%%%%%%%%%%%%%%%%%%%%%%%%
{
\paper{Takens F 1981 in {\bf Dynamical Systems and Turbulence}; Casdagli 1991 in {\bf Physica D}}

\begin{frame}{State Space Reconstruction Theorem}

Uniform time-delay embedding matrix
$\boldsymbol{X}(t) = \{ x(t), x(t-\tau) , ...,x(t - (m-1)\tau  ) \}$ 
defines a map $\Phi: M \rightarrow \mathbb{R}^m$ such that 
\begin{eqnarray*}
\boldsymbol{X}(t) = \Phi(s(t))$
\end{eqnarray*}
where $\Phi$ is a diffeomorphic map whenever $\tau > 0$ 
and $m > 2d_{box}$ and $d_{box}$ is the box-counting dimension of $M$.

\end{frame}
}




\subsection{}
%%%%%%%%%%%%%%%%%%%%%%%%%%%%%%%%%%%%%%%%%%%%%%%%%%%%%%%%
{
\paper{Frank et al. 2010 in {\bf AAAI Conference on Artificial Intelligence} and Sama et al. 2013 in {\bf Neurocomputing} }

\begin{frame}{Uniform Time-Delay Embedding (UTDE)}

For a given discrete time series $\{x_n\}_{n=1}^{N} = [x_1 , x_2, \dots, x_N]$
of sample length $N$, a uniform time-delay embedding matrix is defined as 
\begin{eqnarray*}
 \mathbf{X}^{m}_{\tau}
  = \begin{pmatrix} \nonumber
      \tilde{x}_n  & \\
      \tilde{x}_{n-\tau}  & \\
      \vdots  &  \\
      \tilde{x}_{n-(m-1)\tau} & \\
      \end{pmatrix}^T
\end{eqnarray*}
where $m$ is the \textbf{embedding dimension}  and  $\tau$ is the \textbf{ embedding delay}.


The sample length for $\tilde{x}(n-i\tau)$, where $0 \leq i \leq (m-1)$, is $N-(m-1)\tau$,
and the dimensions of $\mathbf{X}^{m}_{\tau}$ are ($m$,$(N-(m-1)\tau)$).

\end{frame}
}


\subsection{Estimation of Embedding Parameters}
%%%%%%%%%%%%%%%%%%%%%%%%%%%%%%%%%%%%%%%%%%%%%%%%%%%%%%%%
{
\paper{Cao L. 1997 in {\bf Physica D}; Kabiraj et al. 2012 in {\bf Chaos}}

\begin{frame}{Estimation of Embedding Parameters }

\begin{block}{False Nearest Neighbours (FNN) for $m$}
Unfold the attractor (i.e. evolving trajectories
in a state space).
\end{block}

\begin{block}{Average Mutual Information (AMI) for $\tau$}
Maximize the information in the RSSs.
\end{block}
	
\end{frame}
}




\subsection{FNN and AMI}
%%%%%%%%%%%%%%%%%%%%%%%%%%%%%%%%%%%%%%%%%%%%%%%%%%%%%%%%
{
\paper{England 2007, {\bf Gait and Posture}}

\begin{frame}{False Nearest Neighbours (FNN) for embedding dimension}
    \begin{figure}
        \centering
       \includegraphicscopyright[width=0.8\linewidth]{england2007/drawing}{
	Figure 2 England 2007.}
	\caption{
		(A) Plot of $x(t)=\sin(2\pi t) + \cos(2 \pi t)$
		with embedding dimension $n=3$ and (B) $n=2$.
		Note that the false-nearest-neighbor illustrated
		by the intersectoin (-0.25, 0.25) when $n=2$.
	} 
   \end{figure}
	
\end{frame}
}


\subsection{Estimation of Embedding Parameters}
%%%%%%%%%%%%%%%%%%%%%%%%%%%%%%%%%%%%%%%%%%%%%%%%%%%%%%%%
{
\paper{Cao L. 1997 in {\bf Physica D}}

\begin{frame}{False Nearest Neighbours (FNN) for embedding dimension}

\begin{block}{False Nearest Neighbours (FNN)}
\begin{equation*}
E(m) &= \frac{1}{N-m\tau} \sum_{i=1}^{N-m\tau} 
       \frac{ || \boldsymbol{X}_i(m+1) - \boldsymbol{X}_{n(i,m)}(m+1) || }
            { || \boldsymbol{X}_i(m) - \boldsymbol{X}_{n(i,m)}(m) ||  } 
\end{equation*}
\end{block}

\begin{block}{ $E_1(m)$ and $E_2(m)$ }
\begin{equation*}
E_1(m) = \frac{ E(m+1) } { E(m)} \quad 
E_2(m) = \frac{ E^* (m+1) } { E^*(m)}
\end{equation*}
\end{block}
	
\end{frame}
}





\subsection{FNN and AMI}
%%%%%%%%%%%%%%%%%%%%%%%%%%%%%%%%%%%%%%%%%%%%%%%%%%%%%%%%
{
%\paper{Xochicale M 2018, {\bf PhD thesis}}

\begin{frame}{False Nearest Neighbours (FNN) for embedding dimension}
    \begin{figure}
        \centering
        \includegraphicscopyright[width=0.9\linewidth]{nonlinear-analyses/cao}{
	Figure is adapted from Cao L 1997 in {\bf Physica D}}
	\caption{(A,B) $E_1(m)$ and (C, D) $E_2(m)$ values for (E) chaotic 
		and (F) random time series} 
   \end{figure}
	
\end{frame}
}




\subsection{Estimation of Embedding Parameters}
%%%%%%%%%%%%%%%%%%%%%%%%%%%%%%%%%%%%%%%%%%%%%%%%%%%%%%%%
{
\paper{Kantz et al. 2003 in {\bf BOOK: Nonlinear Time Series Analysis}}

\begin{frame}{Average Mutual Information (AMI) for embedding delay}

In order to obtain $\tau_0$, 
"it has to be found in the first minimum of $I(\tau)$ where $x(n+\tau)$ 
adds maximal information to the knowledge from $x(n)$" meaning that the 
redundancy between $x(n+\tau)$ and $x(n)$ is the least .


\end{frame}
}





\subsection{Estimation of Embedding Parameters}
%%%%%%%%%%%%%%%%%%%%%%%%%%%%%%%%%%%%%%%%%%%%%%%%%%%%%%%%
{
\paper{Kabiraj et al. 2012 in {\bf Chaos}}

\begin{frame}{Average Mutual Information (AMI) for embedding delay}

\begin{block}{Average Mutual Information (AMI)}
\begin{equation*}
I(\tau) = \sum_{i,j}^N p_{ij} log_2 \frac{ p_{ij} }{ p_i p_j }.
\end{equation*}
\end{block}

where: 
$p_i$ is the probability that $x(n)$ has a value inside the $i$-th bin of 
the histogram, $p_j$ is the probability that $x(n+\tau)$ has a value inside 
the $j$-th bin of the histogram and $p_{ij}(\tau)$ is the probability 
that $x(n)$ is in bin $i$ and $x(n+\tau)$ is in bin $j$.

\end{frame}
}



\subsection{}
%%%%%%%%%%%%%%%%%%%%%%%%%%%%%%%%%%%%%%%%%%%%%%%%%%%%%%%%
{

\begin{frame}{Average Mutual Information (AMI) for embedding delay}
    \begin{figure}
        \centering
        \includegraphicscopyright[width=0.7\linewidth]{nonlinear-analyses/ami}
	{Figure is adapted from Kabiraj et al. 2012 in {\bf Chaos}}
	\caption{(A, B) AMI values for (C) chaotic and (D) noise time series.} 
   \end{figure}
	
\end{frame}
}








\subsection{RP and RQA}
%%%%%%%%%%%%%%%%%%%%%%%%%%%%%%%%%%%%%%%%%%%%%%%%%%%%%%%%
{
\paper{Pawar et al. 2018}

\begin{frame}{Recurrence Plot (RP)}

    \begin{figure}
        \includegraphicscopyright[width=0.6\linewidth]{pawar2018/drawing}
	{Figure is from Pawar 2018.} 
	\caption{
	The vertical lines in the RP show that more that one point of 
	the same trajectory are recurring at the same time.
	The diagonal lines in the RP depict that two trajectories are running in parallel
	to each other.
}
   
\end{figure}
	


%    \begin{figure}
%        \includegraphicscopyright[width=1.0\linewidth]{nonlinear-analyses/rp}
%		{Figure is adapted from (Marwan et al. 2007)}
%	\caption{(A) State space for Lorenz systems, and 
%		(B) Recurrence plot with embeddings ($m=1$, $\tau=1$) and $\epsilon=5$} 
%   \end{figure}
%


\end{frame}
}



\subsection{}
%%%%%%%%%%%%%%%%%%%%%%%%%%%%%%%%%%%%%%%%%%%%%%%%%%%%%%%%
{
\paper{Eckmann et al. 1987 in {\bf Europhysics Letters}}

\begin{frame}{Recurrence Plot}
    \begin{figure}
        \includegraphicscopyright[width=1.0\linewidth]{nonlinear-analyses/rp}
		{Figure is adapted from (Marwan et al. 2007)}
	\caption{(A) State space for Lorenz systems, and 
		(B) Recurrence plot with embeddings ($m=1$, $\tau=1$) and $\epsilon=5$}. 
   \end{figure}

\end{frame}
}




%%%%%%%%%%%%%%%%%%%%%%%%%%%%%%%%%%%%%%%%%%%%%%%%%%%%%%%%
{
\paper{Eckmann et al. 1987 in {\bf Europhysics Letters}}

\begin{frame}{Recurrence Plots}


$\mathbf{R}^{m}_{i,j} (\epsilon)$ is two dimensional plot of $N \times N$ square matrix
defined by

%%********************************[EQUATION]************************************
\begin{equation*}
\mathbf{R}^{m}_{i,j} (\epsilon) = 
\Theta ( \epsilon_i - \Vert X(i) - X(j) \Vert ), 
\quad i,j=1,\dots,N
\end{equation*}
%%********************************[EQUATION]************************************

where $N$ is the number of considered reconstructed states of $X(i)$
($X(i) \in \mathbb{R}^m$), 
$\epsilon$ is a threshold distance, 
$ \Vert  \cdot \Vert$ a norm, 
and $\Theta( \cdot )$ is the Heaviside function.

\end{frame}
}




\subsection{}
%%%%%%%%%%%%%%%%%%%%%%%%%%%%%%%%%%%%%%%%%%%%%%%%%%%%%%%%
{
\paper{Marwan et al. 2007 in {\bf Physics Reports}}

\begin{frame}{Recurrence Plot Patterns}
    \begin{figure}
        \includegraphicscopyright[width=\linewidth]{nonlinear-analyses/rpsp}{Figure is adapted from (Marwan et al. 2007)}
	\caption{Recurrence plots for (A) uniformly distributed noise,
		(B) super-positionet harmonic oscillation,
		(C) drift logistic map with a linear increase term, and
		(D) disrupted brownian motion.
		} 
   \end{figure}
	
\end{frame}
}


\subsection{}
%%%%%%%%%%%%%%%%%%%%%%%%%%%%%%%%%%%%%%%%%%%%%%%%%%%%%%%%
{
\paper{Marwan and Webber, 2015}

\begin{frame}{Recurrence Quantification Analysis (RQA)}

\begin{description}
\item [ \textbf{REC} ] enumerates the black dots in the RP.
%%********************************[EQUATION]************************************
\begin{equation*}
	REC(\epsilon,N)= 
	\frac{1}{N^2 - N} \sum^{N}_{i \neq j = 1} 
	\mathbf{R}^{m}_{i,j}(\epsilon)
\end{equation*}
%%********************************[EQUATION]************************************
\item [ \textbf{DET} ] fraction of recurrence points that form diagonal lines. \\
			\textit{(interpreted as the predictability where, for example,
				periodic signals show longer diagonal lines 
				than chaotic ones.
				)}
%%********************************[EQUATION]************************************
\begin{equation*}
	DET=\frac{\sum^{N}_{l=d_{min}} l H_D{l} }{\sum^{N}_{i,j=1} 
	\mathbf{R}^{m}_{i,j}(\epsilon) }
\end{equation*}
%%********************************[EQUATION]************************************

\end{description}


	
\end{frame}
}





\subsection{}
%%%%%%%%%%%%%%%%%%%%%%%%%%%%%%%%%%%%%%%%%%%%%%%%%%%%%%%%
{
%\paper{Marwan et al. 2007 in {\bf Physics Reports}}
\paper{Marwan and Webber, 2015}

\begin{frame}{Recurrence Quantification Analysis (RQA)}

\begin{description}
\item [ \textbf{RATIO} ] is the ratio of DET to REC. \\
			\textit{(useful to discover dynamic transitions)}.
\item [ \textbf{ENTR} ] Shannon entropy of the frequency distribution of the 
			diagonal line lengths.
			\textit{(useful to represent the complexity of the 
				structure of the time series)}
%%********************************[EQUATION]************************************
\begin{equation*}
	ENT= - \sum^{N}_{l=d_{min}} p(l) ln p(l),
\end{equation*}
%%********************************[EQUATION]************************************
where 
%%********************************[EQUATION]************************************
\begin{equation*}
	p(l)=\frac{ H_D(l) }{ \sum^{N}_{ l=d_{min} } H_D(l) }
\end{equation*}
%%********************************[EQUATION]************************************

\end{description}


	
\end{frame}
}




\subsection{}
%%%%%%%%%%%%%%%%%%%%%%%%%%%%%%%%%%%%%%%%%%%%%%%%%%%%%%%%
{
\paper{Xochicale 2019 in {\bf PhD thesis}}

\begin{frame}{RQA}

    \begin{figure}
        \includegraphicscopyright[width=0.5\linewidth]{nonlinear-analyses/rqas}{Figure is adapted from Xochicale 2019}
	\caption{Recurrence Quantification Analysis with \\
		$m_0=6$, $\tau_0=8$ and $\epsilon=1$.
		} 
   \end{figure}
		
\end{frame}
}







\subsection{}
%%%%%%%%%%%%%%%%%%%%%%%%%%%%%%%%%%%%%%%%%%%%%%%%%%%%%%%%
{
\paper{Xochicale 2019 in {\bf PhD thesis}}

\begin{frame}{3D surface plots of RQA}

    \begin{figure}
        \includegraphicscopyright[width=0.5\linewidth]{nonlinear-analyses/fig_3_07_a}{Figure is adapted from Xochicale 2019}
	\caption{(A) 3D surface plots for 
	with increasing pair of embedding parameters 
	($0 \le m \le 10$, $0 \le \tau \le 10$) and $\epsilon=3.0$.
	(B) Surface plot A with decimal increase of 0.1 
	for recurrence thresholds ($ 0.2 \ge \epsilon \le 3 $).
	} 
   \end{figure}
		
\end{frame}
}



\subsection{}
%%%%%%%%%%%%%%%%%%%%%%%%%%%%%%%%%%%%%%%%%%%%%%%%%%%%%%%%
{
\paper{Xochicale 2019 in {\bf PhD thesis}}

\begin{frame}{3D surface plots of RQA}

    \begin{figure}
        \includegraphicscopyright[width=0.9\linewidth]{nonlinear-analyses/fig_3_07_b}{Figure is adapted from Xochicale 2019}
	\caption{3D surface plots for 
		(A) uniformly distributed noise,
		(B) super-positionet harmonic oscillation,
		(C) drift logistic map with a linear increase term,
		(D) disrupted brownian motion, and
		(E) Lorenz system.
		} 
   \end{figure}
		
\end{frame}
}





